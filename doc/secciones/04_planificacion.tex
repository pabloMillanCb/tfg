\chapter{Planificación}

\section{Metodología utilizada}

\subsection{Metodologías de desarrollo ágil}
El software es un sistema complejo de crear que depende de muchas partes interconectadas. No es igual de complejo programar un pequeño \textit{script} que ordene los ficheros de un directorio que construir una aplicación de cierta envergadura que responda a las necesidades de un usuario o un cliente concreto, mucho más si además se debe coordinar un equipo de personas en la concepción del mismo. Existe un gran factor de riesgo e incertidumbre, sobre todo al inicio de un proyecto. Las metodologías ágiles son una forma de desarrollar software enfocada a equipos pequeños que busca paliar estos problemas a la hora de encarar un proyecto. El movimiento ágil comenzó oficialmente en 2001 con la creación del Manifiesto Ágil. Este escrito fue firmado por 17 autores y daban nombre al método que ellos usaban para desarrollar software junto con una serie de valores por los que se regían, como por ejemplo que valoraban:

\begin{itemize}
    \item Individuos e interacciones por encima de procesos y herramientas.
    \item Software funcional por encima de documentación detallada.
    \item Colaboración con el cliente en lugar de negociación de contratos.
    \item Responder a cambios mientras se sigue un plan.
\end{itemize}

\section{Temporización}

\section{Seguimiento del desarrollo}
