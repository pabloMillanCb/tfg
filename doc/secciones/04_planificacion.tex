\chapter{Planificación}

\section{Metodología utilizada}

El software es un sistema complejo de crear que depende de muchas partes interconectadas. No es igual de dificultoso programar un pequeño \textit{script} que ordene los ficheros de un directorio de un sistema operativo que construir una aplicación de cierta envergadura que responda a las necesidades de un usuario o un cliente concreto, mucho más si además se debe coordinar un equipo de personas en la concepción del mismo. Existe un gran factor de riesgo e incertidumbre, sobre todo al inicio de un proyecto. Las \textbf{metodologías de desarrollo ágil} son una forma de desarrollar software enfocada a equipos pequeños que busca paliar estos problemas a la hora de encarar un proyecto,. En otras metodologías anteriores se redactaban unos requisitos completos y fijos al inicio del proyecto y antes si quiera de comenzar la implementación. Esto era una fuente común de conflictos, ya que es muy común que los requisitos cambien a lo largo del desarrollo. Quizás el cliente cambie de idea, o el equipo encuentre mejores formas de realizar una tarea, o el contexto en el ámbito en el que se desarrolla el producto ha cambiado y hay que adaptar ciertas partes del proyecto, etc. La incertidumbre es un elemento muy presente, y las metodologías ágiles la abraza. Conviven con el cambio inevitable estableciendo unas bases menos sólidas pero más flexibles. Esta forma de trabajar se enunció oficialmente por primera vez en 2001, en el documento de \textbf{Manifiesto Ágil}\footnote{El Manifiesto Ágil se puede consultar en esta web oficial. Hay una versión en español en la misma web. \url{https://agilemanifesto.org/}}. Esta forma de organizar proyectos  se fundamenta en cuatro puntos, extraidos del libro \textit{Métodos Ágiles y Scrum}\cite{agilescrum}:

\begin{itemize}
    \item \textbf{Valorar a individuos y sus interacciones}, frente a procesos y herramientas. Aunque todas las ayudas para desarrollar un trabajo son importantes, nada sustituye a las personas, a las que hay que dar toda la importancia y poner en primer plano.
    \item \textbf{Valorar más el software (producto) que funciona}, que una documentación exhaustiva. Porque había llegado un punto en el que documentar el trabajo había alcanzado tanta importancia como el objeto del trabajo: el producto.
    \item \textbf{Valorar más la colaboración con el cliente} que la negociación de un contrato. La forma más productiva de sacar adelante un trabajo es establecer un marco de confianza y colaboración con quien nos lo encarga. Sin embargo se estaba poniendo el foco en cerrar un contrato atado que sirviera ante todo como una herramienta de protección, como si el cliente y equipo fueran dos partes enfrentadas, cuando en realidad comparten objetivos e intereses.
    \item \textbf{Valorar más la respuesta al cambio} que el seguimiento de un plan. Se trata de apreciar la incertidumbre como un componente básico del trabajo, por lo que la adaptación y la flexibilidad se convierten en virtudes y no en amenazas. El seguimiento ciego de un plan lleva, salvo contadas excepciones, al fracaso si no se puede corregir la dirección ante los inevitables cambios que van surgiendo.
\end{itemize}

Este típo de metodologías intentan aplicar un cambio de paradigma: mientras que tradicionalmente se fijaban unos requisitos a partir de los cuales se estimaban los recursos y plazos que se requerirían para su terminación, la forma ágil sería fijar unos recursos y fechas disponibles en las que se disponen para el trabajo, y a partir de ahí estimar los requisitos que va a tener el producto al final. Se eligió optar por esta forma de encarar proyectos ya que en primer lugar el trabajo tenía una fecha de finalización fíja, y además no se tenía experiencia en algunos de los campos requeridos para el desarrollo, como podría ser por ejemplo la implementación de interfaces web. Se iba a requerir de un periodo de formación paralela al desarrollo, lo cual con total probabilidad cambiará la visión del desarrollador sobre la estimación de los costes y tiempo del proyecto. Sería contraproducente espablecer desde un inicio unos requisitos fijos si se prevé que van a cambiar, y  las metodologías ágiles se ajustaban perfectamente a este escenario.

\section{Temporización}

\section{Seguimiento del desarrollo}
