\thispagestyle{empty}

\begin{center}
{\large\bfseries EditAR \\ Editor 3D web para escenas de Realidad Aumentada }\\
\end{center}
\begin{center}
Pablo Millán Cubero\\
\end{center}

%\vspace{0.7cm}

\vspace{0.5cm}
\noindent\textbf{Palabras clave}: \textit{software libre, informática gráfica, Typescript, React, Firebase, NodeJS, Express, Android Studio, Kotlin, Sceneview, GLB}
\vspace{0.7cm}

\noindent\textbf{Resumen}\\
Las tecnologías de Realidad Aumentada nos permiten combinar elementos virtuales y reales en tiempo real pudiendo así crear experiencias aplicables a muchos campos; filtros con maquillaje para fotos en redes sociales, previsualizar cómo quedaría un mueble en tu propio salón, dibujar en un partido de fútbol la trayectoria que recorrió la pelota en una repetición, o incluso en videojuegos como Pokémon GO.

Con el avance de los teléfonos móviles de todas las gamas en cuanto a cámaras y capacidad de procesamiento en la última década estas experiencias han pasado a estar al alcance de todo el mundo que posea un dispositivo básico con resultados muy vistosos. Sin embargo, en la mayoría de experiencias de Realidad Aumentada el usuario ocupa el rol de consumidor pasivo del contenido que generan los desarrolladores.

Se propone entonces desarrollar una aplicación web en el que el usuario pueda cargar múltiples modelos 3D con textura y animaciones, aplicarles a estos transformaciones como translaciones, rotaciones y escalado, reproducir animaciones y cargar pistas de audio. Todo ello desde un sencillo e intuitivo editor que puede usar cualquiera en el que no haga falta tener conocimientos previos. La escena 3D que se cree podrá posteriormente descargarse o guardarse en un servidor web asociado a una cuenta de usuario. El usuario podrá además, desde una aplicación Android cargar sus escenas y reproducirlas, pudiendo visualizarlas en su entorno a través de la cámara del dispositivo móvil.



\cleardoublepage

\begin{center}
	{\large\bfseries EditAR \\ Web editor for Augmented Reality scenes}\\
\end{center}
\begin{center}
	Pablo Millán Cubero\\
\end{center}
\vspace{0.5cm}
\noindent\textbf{Keywords}: \textit{software libre, informática gráfica, Typescript, React, Firebase, NodeJS, Express, Android Studio, Kotlin, Sceneview, GLB}
\vspace{0.7cm}

\noindent\textbf{Abstract}\\
Augmented Reality technologies let us merge virtual elements with camera footage in real time, letting us create experiences for many fields; makeup filters for social network photos, preview new furniture for your livingroom, draw the trayectory of the ball in a football match replay, or even in videogames like Pokemon GO.

With the improvement of smartphone's cameras and processing power in the last decade, this experiences are know within everybody's reach if you have a basic device. However, the mayority of Augmented Reality experiences are based in the pasive comsumption of content pregenerated by developers.

The proposed web application lets the user load multiple 3D models with textures and animations, apply transformations like translations, rotations and scales, and play audio tracks. All this in an easy and intuitive editor that anyone can use without previous knowledge. The resultant 3D scene can be either downloaded or uploaded to a web server asociated to an user account. Aditionaly, the user can load the created scenes within an Android app and play them with their device's camera, placing it in their environment.


\cleardoublepage

\thispagestyle{empty}

\noindent\rule[-1ex]{\textwidth}{2pt}\\[4.5ex]

D. \textbf{Tutora/e(s)}, Profesor(a) del ...

\vspace{0.5cm}

\textbf{Informo:}

\vspace{0.5cm}

Que el presente trabajo, titulado \textit{\textbf{Chief}},
ha sido realizado bajo mi supervisión por \textbf{Estudiante}, y autorizo la defensa de dicho trabajo ante el tribunal
que corresponda.

\vspace{0.5cm}

Y para que conste, expiden y firman el presente informe en Granada a Junio de 2018.

\vspace{1cm}

\textbf{El/la director(a)/es: }

\vspace{5cm}

\noindent \textbf{(nombre completo tutor/a/es)}

\chapter*{Agradecimientos}

A mis padres por permitirme estudiar esta carrera en Granada y apoyarme.

A mis abuelos, en especial a mi abuela Elena, que tanto presume de "su nieto el ingeniero".

A mis tíos "los franceses".

A mis amigos del Krustáceo Krujiente, Juan, Antonio, y mis compañeros de trinchera Pepe, Finn, Moisés y Pablo.



